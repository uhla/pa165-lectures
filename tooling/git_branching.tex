\section[Git Branching]{Git Branching}
\subsection[]{Overview}
\begin{frame}
\frametitle{Git Branching - Overview}
\begin{itemize}
	\item A branch represents an independent line of development.
	\item Lightweight implementation of branching – Git stores a branch as \textbf{a reference to a commit}.
	\item Keeps history as a tree, where \textbf{each commit is a node} in the tree, and has one or more parents.
	\item History is \textbf{extrapolated through the commit relationships}.
	\item It’s a good practice to \textbf{spawn a new branch to encapsulate your changes} no matter how big the changes are.
\end{itemize}
\end{frame}



\subsection[]{Local Branches}
\begin{frame}
\frametitle{Git Branching - Local Branches}
\begin{itemize}
	\item \textbf{Non-tracking local branches}
		\begin{itemize}
		\item Exist on user’s machine.
		\item Not associated with any other branch.
		\item User needs to specify upstream branch when running push or pull commands.
		\end{itemize}
	\item \textbf{Tracking local branches}
		\begin{itemize}
		\item Exist on user’s machine.
		\item Tracking branch is a branch that has a direct relationship to another branch.
		\item Local tracking branches in most cases track a remote tracking branch.
		\item Allow user to run git pull and git push without specifying which upstream branch to use.
		\end{itemize}
\end{itemize}
\end{frame}

\subsection[]{Local Branches - remote-tracking branches}
\begin{frame}
\frametitle{Git Branching - remote-tracking branches}
\begin{itemize}
	\item \textbf{Remote}
		\begin{itemize}
		\item Remote connection (bookmark) into other repository.
		\end{itemize}
	\item \textbf{Remote branch}
		\begin{itemize}
		\item Branch on a remote location.
		\end{itemize}
	\item \textbf{Remote-tracking branch}
		\begin{itemize}
		\item Local cache for what the remote repositories contain.
		\item (remote)/(branch)
			\begin{itemize}
			\item origin/master
			\item origin/test-branch
			\end{itemize}
		\end{itemize}
	\item \textbf{Note:}
		\begin{itemize}
		\item “origin” and “master” are not special.
		\end{itemize}
\end{itemize}
\end{frame}

\subsection[]{Git Branching - merge}
\begin{frame}
\frametitle{Git Branching - merge}
\begin{itemize}
	\item Way of putting a forked \textbf{history back together} again.
	\item\textbf{Non-destructive} operation.
	\item All the operations always \textbf{merge into the current} branch.
	\item Git has \textbf{several distinct algorithms} to accomplish the merge.
\end{itemize}
Note:
\begin{itemize}
	\item git pull command effectively runs git fetch and git merge.
\end{itemize}
\end{frame}

\begin{frame}
\frametitle{Git Branching - merge}
\begin{itemize}
	\item \textbf{3-Way Merge}
		\begin{itemize}
		\item Creates \textbf{merge commit} that ties together the histories of both branches.
		\item Merge commit as a \textbf{symbolic joining} of the two branches.
		\item Original \textbf{context is maintained}.
		\end{itemize}
	\item \textbf{Fast-Forward Merge}
		\begin{itemize}
		\item Requires \textbf{linear path} from the current branch tip to the target branch.
		\item Usually \textbf{facilitated through rebasing} – suitable for small tasks and fixes.
		\item Context of the affected commits as part of an earlier feature branch is lost.
		\end{itemize}
\end{itemize}
\end{frame}

\subsection[]{Git Branching - 3-way merge}
\begin{frame}
\frametitle{Git Branching - 3-way merge}
\begin{tikzpicture}
  \node (img1) {\includegraphics[width=6.4cm, height=4cm]{3waymerge-1.png}};
%  \pause
  \begin{scope}[on background layer]
  	\node (img2) at (img1.south east) [yshift=-1cm] {\includegraphics[width=7.7cm, height=4cm]{3waymerge-2.png}};
  \end{scope}
\end{tikzpicture}
\end{frame}

\subsection[]{Git Branching - fast-forward merge}
\begin{frame}
\frametitle{Git Branching - fast-forward merge}
\begin{tikzpicture}
  \node (ffwd1) {\includegraphics[width=8.18cm, height=3cm]{fastfwd-1.png}};
%  \pause
  \begin{scope}[on background layer]
  	\node (ffwd2) at (ffwd1.south east) [yshift=-1.5cm, xshift=-2cm] {\includegraphics[width=7.9cm, height=3cm]{fastfwd-2.png}};
  \end{scope}
\end{tikzpicture}
\end{frame}

\subsection[]{Git Branching - rebase}
\begin{frame}
\frametitle{Git Branching - rebase}
\begin{itemize}
	\item Process of \textbf{moving or combining} a sequence of commits to a new base commit – alternative to merge.
	\item Makes the branch appear as if you'd created it from a different commit.
	\item Git takes changes from your branch and \textbf{replays them} on top of the destination branch.
	\item Result branch \textbf{looks the same} but it's composed of entirely new commits.
	\item Do not rebase commits that exist \textbf{outside your repository} (unless you have a good reason to do so).
\end{itemize}
\end{frame}

\begin{frame}
\frametitle{Git Branching - rebase}
\includegraphics[width=\textwidth, height=0.55\textwidth]{rebase.png}
\end{frame}

\begin{frame}
\frametitle{Git Branching - rebase}
\begin{itemize}
	\item Rebase helps to maintain a \textbf{linear project history} – that allows an easier investigation of regression issues.
	\item \textbf{Workflow example:}
		\begin{enumerate}
		\item User creates the new task branch from master and starts working in it.
		\item There is an active development on a master branch.
		\item User wants to get the latest updates from master to task branch.
		\item User performs regular rebase operation to move his commits on top of latest master commits.
		\item User is done with his task and performs the final rebase and merge to master.
		\item Git is able to apply fast-forward algorithm for the merge.
		\end{enumerate}
\end{itemize}
\end{frame}

\begin{frame}
\frametitle{Git Branching - rebase}
\begin{itemize}
	\item \textbf{Interactive rebase}
		\begin{itemize}
		\item Allows user to \textbf{alter individual commits} in the process of rebasing.
		\item Support for powerful \textbf{history rewriting} features.
		\item Useful for history cleanup: reword, edit, squash, fixup
		\item Always amend commits that have \textbf{not been pushed} yet to avoid confusion.
		\end{itemize}
\end{itemize}
\end{frame}

\subsection[]{Git Branching - conflicts}
\begin{frame}
\frametitle{Git Branching - conflicts}
\begin{itemize}
	\item Conflicts may occur during merge and rebase operations.
	\item Use the suitable \textbf{merge strategy} to avoid conflicts.
	\item When the conflict occurs:
		\begin{itemize}
		\item \textbf{Abort} the merge with.
		\item \textbf{Work through} the conflict and \textbf{continue}.
		\end{itemize}
	\item \textbf{Visualize and resolve} conflict in merge tool.
\end{itemize}
Note:
\begin{itemize}
	\item Rebase has an option to \textbf{skip/bypass} conflicting commit.
\end{itemize}
\end{frame}

\subsection[]{Git Branching - commands}
\begin{frame}
\frametitle{Git Branching - commands}
\textbf{git branch}
	\begin{itemize}
	\item List all of the branches in your repository.
	\end{itemize}
\textbf{git branch <new-branch-name>}
	\begin{itemize}
	\item Create a new branch called <new-branch-name>.
	\end{itemize}
\textbf{git branch -d <existing-branch-name>}
	\begin{itemize}
	\item Delete the existing branch, safely. Use –D to force it.
	\end{itemize}
\textbf{git checkout <existing-branch-name>}
	\begin{itemize}
	\item Navigate between the existing branches.
	\end{itemize}
\textbf{git checkout -b <new-branch-name> <remote>/<branch-name>}
	\begin{itemize}
	\item Create and checkout a new local tracking branch.
	\item Or simply use the previous command if there is only one remote-tracking branch called <existing-branch-name>. This applies to Git 1.6.6+. 
	\end{itemize}
\end{frame}


\begin{frame}
\frametitle{Git Branching - commands}
\textbf{git remote}
	\begin{itemize}
	\item List the connections to remote repositories. Use -v to get URLs. 
	\end{itemize}
\textbf{git remote add <remote-name> <url>}
	\begin{itemize}
	\item Add a new connection to a remote repository.
	\end{itemize}
\textbf{git remote rm <remote-name>}
	\begin{itemize}
 	\item Remove the connection to a remote repository.
	\end{itemize}
\textbf{git fetch <remote-name>}
	\begin{itemize}
	\item Fetch all of the branches from the remote repository.
	\end{itemize}
\textbf{git pull <remote-name>}
	\begin{itemize}
	\item Fetch the remote’s copy of the current branch and merge it into the local copy.
	\end{itemize}
\textbf{git push <remote-name> <branch-name>}
	\begin{itemize}
	\item Push the specified branch to remote repository
	\end{itemize}
\end{frame}

\begin{frame}
\frametitle{Git Branching - commands}
\textbf{git merge <existing-branch-name>}
	\begin{itemize}
	\item Merge the specified branch into current one and let Git to choose an algorithm.
	\end{itemize}
\textbf{git merge -{}-no-ff <existing-branch-name>}
	\begin{itemize}
	\item Merge the specified branch into current one and generate merge commit.
	\end{itemize}
\textbf{git merge -{}-ff-only <existing-branch-name>}
	\begin{itemize}
	\item Merge the specified branch into current one and refuse to merge when fast-forward is not possible.
	\end{itemize}
\textbf{git checkout task} \\
\textbf{git rebase master}
	\begin{itemize}
	\item Move the entire task branch to begin on the tip of the master branch, effectively incorporating all of the new commits in master.
	\end{itemize}
\end{frame}