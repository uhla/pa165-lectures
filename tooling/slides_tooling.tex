\documentclass[pdf,compress]{beamer}
\mode<presentation>{}
\usepackage{lmodern}
\usepackage[normalem]{ulem}
\usepackage[final]{pdfpages}
\usepackage[czech]{babel}
\usepackage[T1]{fontenc}
\usepackage[utf8x]{inputenc}

\usepackage{moreverb}

\addto\captionsczech{% Replace "english" with the language you use
  \renewcommand{\contentsname}%
    {Contents}%
    }

\usetheme[subsection=false]{Dresden}


\definecolor{SWIOrangeDark}{RGB}{249, 157, 28} % SWI logo orange color
\definecolor{SWIGray}{RGB}{68, 68, 68} % SWI logo gray color
\definecolor{SWIOrangeLight}{RGB}{251, 188, 102} % lighter orange

\setbeamercolor{palette primary}{bg=SWIOrangeDark,fg=SWIGray}
\setbeamercolor{palette secondary}{bg=SWIOrangeDark,fg=SWIGray}
\setbeamercolor{palette tertiary}{bg=SWIOrangeDark,fg=SWIGray}
\setbeamercolor{palette quaternary}{bg=SWIOrangeDark,fg=SWIGray}
\setbeamercolor{structure}{fg=SWIOrangeDark} % itemize, enumerate, etc
\setbeamercolor{section in toc}{fg=SWIOrangeDark} % TOC sections

% Override palette coloring with secondary
\setbeamercolor{subsection in head/foot}{bg=SWIOrangeLight,fg=white}

\setbeamertemplate{footline}
{
  \leavevmode%
  \hbox{%
    \begin{beamercolorbox}[wd=.8\paperwidth,ht=2.25ex,dp=1ex,left]{bg=black,fg=title in head/foot}%
\hspace*{3ex}    \insertframenumber{} / \inserttotalframenumber\hspace*{1ex}
  \end{beamercolorbox}%
  \begin{beamercolorbox}[wd=.2\paperwidth,ht=2.25ex,dp=1ex,center]{bg=black,fg=title in head/foot}%
\includegraphics[height=0.5cm]{../shared_logo/SW_Logo_md.png} \hspace*{1ex}%
  \end{beamercolorbox}%

  }
  \vskip0pt%
}%

\AtBeginSection[]
{
\setbeamercolor{section in toc}{fg=SWIOrangeDark}
\setbeamercolor{section in toc shaded}{fg=structure}
\begin{frame}<beamer>
  \frametitle{Contents}
  \tableofcontents[currentsection]
\end{frame}
}

\makeatother
\makeatletter

\setbeamertemplate{navigation symbols}{}

\title{Tooling for Java EE applications}
\subtitle{PA165}
\date{26.\,9.\,2017}
\author{Jiří Uhlíř, Martin Kotala}
%\institute{Fakulta informatiky, Masarykova univerzita}

\begin{document}
\frame{\titlepage}

\section[]{}
\begin{frame}
\frametitle{Contents}
\tableofcontents
\end{frame}


\section[Maven]{Maven}
\subsection[]{History}
\begin{frame}
\frametitle{History}
\begin{itemize}
\item Usnadnění tvorby příkladů pro použití v testech a domácích úkolech
\item Generování sad otázek pro odpovědníky
\end{itemize}
\end{frame}

\subsection[]{Comparison with other tools}
\begin{frame}
\frametitle{Comparison with other tools}
\begin{itemize}
\item Jednotná aplikace pro generování příkladů
\item Opravy chyb a vylepšení algoritmů existujících generátorů \pause
\item Generátor příkladů pro algoritmus C-Y-K \pause
\item Rozšiřitelnost vzniklé aplikace pro další typy příkladů (i mimo okruh formálních jazyků) \pause
\item Odlehčení GUI \pause
\item Lokalizace do češtiny a angličtiny
\end{itemize}

\end{frame}

\section[GIT Basics]{GIT Basics}
\subsection[]{Another sub}
\begin{frame}
\frametitle{Cíle práce}
\begin{itemize}
\item Jednotná aplikace pro generování příkladů
\item Opravy chyb a vylepšení algoritmů existujících generátorů \pause
\item Generátor příkladů pro algoritmus C-Y-K \pause
\item Rozšiřitelnost vzniklé aplikace pro další typy příkladů (i mimo okruh formálních jazyků) \pause
\item Odlehčení GUI \pause
\item Lokalizace do češtiny a angličtiny
\end{itemize}

\end{frame}

\section[Another]{Another}
\subsection[]{Another sub}
\begin{frame}
\frametitle{Cíle práce}
\begin{itemize}
\item Jednotná aplikace pro generování příkladů
\item Opravy chyb a vylepšení algoritmů existujících generátorů \pause
\item Generátor příkladů pro algoritmus C-Y-K \pause
\item Rozšiřitelnost vzniklé aplikace pro další typy příkladů (i mimo okruh formálních jazyků) \pause
\item Odlehčení GUI \pause
\item Lokalizace do češtiny a angličtiny
\end{itemize}

\end{frame}

\end{document}
