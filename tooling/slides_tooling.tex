\documentclass[pdf,compress]{beamer}
\mode<presentation>{}
\usepackage{lmodern}
\usepackage[normalem]{ulem}
\usepackage[final]{pdfpages}
\usepackage[czech]{babel}
\usepackage[T1]{fontenc}
\usepackage[utf8x]{inputenc}
\usepackage{tikz} % for images positioning
\usetikzlibrary{backgrounds}

\usepackage{moreverb}

\addto\captionsczech{% Replace "english" with the language you use
  \renewcommand{\contentsname}%
    {Contents}%
    }


%apply styling
%use with \input at the beginning of all presentations

\usetheme[subsection=false]{Dresden}

\setbeamertemplate{navigation symbols}{}

\definecolor{SWIOrangeDark}{RGB}{249, 157, 28} % SWI logo orange color
\definecolor{SWIGray}{RGB}{68, 68, 68} % SWI logo gray color
\definecolor{SWIOrangeLight}{RGB}{251, 188, 102} % lighter orange
\definecolor{SWIGreyLight}{RGB}{108, 108, 108} % lighter grey

\setbeamercolor{palette primary}{bg=SWIGray,fg=SWIOrangeDark}
\setbeamercolor{palette secondary}{bg=SWIGray,fg=SWIOrangeDark}
\setbeamercolor{palette tertiary}{bg=SWIGray,fg=SWIOrangeDark}
\setbeamercolor{palette quaternary}{bg=SWIGray,fg=SWIOrangeDark}
\setbeamercolor{structure}{fg=SWIOrangeDark} % itemize, enumerate, etc
\setbeamercolor{section in toc}{fg=SWIOrangeDark} % TOC sections

% Override palette coloring with secondary
\setbeamercolor{subsection in head/foot}{bg=SWIGreyLight,fg=white}

\setbeamertemplate{footline}
{
  \leavevmode%
  \hbox{%
    \begin{beamercolorbox}[wd=.8\paperwidth,ht=2.25ex,dp=1ex,left]{bg=black,fg=title in head/foot}%
\hspace*{3ex}    \insertframenumber{} / \inserttotalframenumber\hspace*{1ex}
  \end{beamercolorbox}%
  \begin{beamercolorbox}[wd=.2\paperwidth,ht=2.25ex,dp=1ex,center]{bg=black,fg=title in head/foot}%
\includegraphics[height=0.5cm]{../shared/SW_Logo_md.png} \hspace*{1ex}%
  \end{beamercolorbox}%

  }
  \vskip0pt%
}%


\AtBeginSection[]
{
\setbeamercolor{section in toc}{fg=SWIOrangeDark}
\setbeamercolor{section in toc shaded}{fg=structure}
\begin{frame}<beamer>
  \frametitle{Contents}
  \tableofcontents[currentsection, hideallsubsections]
\end{frame}
}

\makeatother
\makeatletter


\title{Tooling for Java EE applications}
\subtitle{PA165}
\date{26.\,9.\,2017}
\author{Jiří Uhlíř, Martin Kotala}
%\institute{Fakulta informatiky, Masarykova univerzita}

\begin{document}
\frame{\titlepage}

\section[]{}
\begin{frame}
\frametitle{Contents}
\tableofcontents[hideallsubsections]
\end{frame}


\section[Maven]{Maven}
\subsection[]{History}
\begin{frame}
\frametitle{History}
\begin{itemize}
\item Usnadnění tvorby příkladů pro použití v testech a domácích úkolech
\item Generování sad otázek pro odpovědníky
\end{itemize}
\end{frame}

\subsection[]{Comparison with other tools}
\begin{frame}
\frametitle{Comparison with other tools}
\begin{itemize}
\item Jednotná aplikace pro generování příkladů
\item Opravy chyb a vylepšení algoritmů existujících generátorů \pause
\item Generátor příkladů pro algoritmus C-Y-K \pause
\item Rozšiřitelnost vzniklé aplikace pro další typy příkladů (i mimo okruh formálních jazyků) \pause
\item Odlehčení GUI \pause
\item Lokalizace do češtiny a angličtiny
\end{itemize}

\end{frame}

\section[GIT Basics]{GIT Basics}
\subsection[]{Another sub}
\begin{frame}
\frametitle{Cíle práce}
\begin{itemize}
\item Jednotná aplikace pro generování příkladů
\item Opravy chyb a vylepšení algoritmů existujících generátorů \pause
\item Generátor příkladů pro algoritmus C-Y-K \pause
\item Rozšiřitelnost vzniklé aplikace pro další typy příkladů (i mimo okruh formálních jazyků) \pause
\item Odlehčení GUI \pause
\item Lokalizace do češtiny a angličtiny
\end{itemize}

\end{frame}

\section[Another]{Another}
\subsection[]{Another sub}
\begin{frame}
\frametitle{Cíle práce}
\begin{itemize}
\item Jednotná aplikace pro generování příkladů
\item Opravy chyb a vylepšení algoritmů existujících generátorů \pause
\item Generátor příkladů pro algoritmus C-Y-K \pause
\item Rozšiřitelnost vzniklé aplikace pro další typy příkladů (i mimo okruh formálních jazyků) \pause
\item Odlehčení GUI \pause
\item Lokalizace do češtiny a angličtiny
\end{itemize}
\end{frame}

\section[Git Branching]{Git Branching}
\subsection[]{Overview}
\begin{frame}
\frametitle{Git Branching - Overview}
\begin{itemize}
	\item A branch represents an independent line of development.
	\item Lightweight implementation of branching – Git stores a branch as \textbf{a reference to a commit}.
	\item Keeps history as a tree, where \textbf{each commit is a node} in the tree, and has one or more parents.
	\item History is \textbf{extrapolated through the commit relationships}.
	\item It’s a good practice to \textbf{spawn a new branch to encapsulate your changes} no matter how big the changes are.
\end{itemize}
\end{frame}



\subsection[]{Local Branches}
\begin{frame}
\frametitle{Git Branching - Local Branches}
\begin{itemize}
	\item \textbf{Non-tracking local branches}
		\begin{itemize}
		\item Exist on user’s machine.
		\item Not associated with any other branch.
		\item User needs to specify which upstream branch when running push or pull commands.
		\end{itemize}
	\item \textbf{Tracking local branches}
		\begin{itemize}
		\item Exist on user’s machine.
		\item Tracking branch is a branch that has a direct relationship to another branch.
		\item Local tracking branches in most cases track a remote tracking branch.
		\item Allow user to run git pull and git push without specifying which upstream branch to use.
		\end{itemize}
\end{itemize}
\end{frame}

\subsection[]{Local Branches - remote-tracking branches}
\begin{frame}
\frametitle{Git Branching - remote-tracking branches}
\begin{itemize}
	\item \textbf{Remote}
		\begin{itemize}
		\item Remote connection (bookmark) into other repository.
		\end{itemize}
	\item \textbf{Remote branch}
		\begin{itemize}
		\item Branch on a remote location.
		\end{itemize}
	\item \textbf{Remote-tracking branch}
		\begin{itemize}
		\item Local cache for what the remote repositories contain.
		\item (remote)/(branch)
			\begin{itemize}
			\item origin/master
			\item origin/test-branch
			\end{itemize}
		\end{itemize}
	\item \textbf{Note:}
		\begin{itemize}
		\item “origin” and “master” are not special.
		\end{itemize}
\end{itemize}
\end{frame}

\subsection[]{Git Branching - merge}
\begin{frame}
\frametitle{Git Branching - merge}
\begin{itemize}
	\item Way of putting a forked \textbf{history back together} again.
	\item\textbf{Non-destructive} operation.
	\item All the operations always \textbf{merge into the current} branch.
	\item Git has \textbf{several distinct algorithms} to accomplish the merge.
\end{itemize}
Note:
\begin{itemize}
	\item git pull command effectively runs git fetch and git merge.
\end{itemize}
\end{frame}

\begin{frame}
\frametitle{Git Branching - merge}
\begin{itemize}
	\item \textbf{3-Way Merge}
		\begin{itemize}
		\item Creates \textbf{merge commit} that ties together the histories of both branches.
		\item Merge commit as a \textbf{symbolic joining} of the two branches.
		\item Original \textbf{context is maintained}.
		\end{itemize}
	\item \textbf{Fast-Forward Merge}
		\begin{itemize}
		\item Requires \textbf{linear path} from the current branch tip to the target branch.
		\item Usually \textbf{facilitated through rebasing} – suitable for small tasks and fixes.
		\item Context of the affected commits as part of an earlier feature branch is lost.
		\end{itemize}
\end{itemize}
\end{frame}

\subsection[]{Local Branches - 3-way merge}
\begin{frame}
\frametitle{Git Branching - 3-way merge}
\begin{tikzpicture}
  \node (img1) {\includegraphics[width=6.4cm, height=4cm]{3waymerge-1.png}};
%  \pause
  \begin{scope}[on background layer]
  	\node (img2) at (img1.south east) [yshift=-1cm] {\includegraphics[width=7.7cm, height=4cm]{3waymerge-2.png}};
  \end{scope}
\end{tikzpicture}


\end{frame}



\end{document}
