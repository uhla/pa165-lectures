\section[Git Basics]{Git Basics}
\subsection[]{Version control}
\begin{frame}
\frametitle{Version control}
\begin{itemize}
	\item Motivation
	\item History
		\begin{itemize}
		\item One file at a time
		\item Centralized (CVS, Subversion)
		\item Distributed (Git, Mercurial)
		\end{itemize}
\end{itemize}
\end{frame}

\subsection[]{Git history}
\begin{frame}
\frametitle{Git history}
\begin{itemize}
	\item Created in 2005 by Linus Torvalds 
		\begin{itemize}
		\item described by himself as "stupid content tracker"
		\item Originally created for linux kernel development
		\end{itemize}
	\item Inspired by BitKeeper, aiming to be performant and free
	\item CVS taken as example of what \textit{not to do}
	\item git - no exact meaning
		\begin{itemize}
		\item random three-letter combination that is pronounceable, and not actually used by any common UNIX command. The fact that it is a mispronunciation of "get" may or may not be relevant.
		\item "global information tracker": you're in a good mood, and it actually works for you. Angels sing, and a light suddenly fills the room.
		\item "g*dd*mn idiotic truckload of sh*t": when it breaks
		\item \url{https://github.com/git/git/blob/master/README.md} 
		\end{itemize}	
\end{itemize}
\end{frame}



\subsection[]{Git characteristicsl}
\begin{frame}
\frametitle{Git characteristics}
\begin{itemize}
	\item Strong support for non-linear development
	%Git supports rapid branching and merging, and includes specific tools for visualizing and navigating a non-linear development history. In Git, a core assumption is that a change will be merged more often than it is written, as it is passed around to various reviewers. In Git, branches are very lightweight: a branch is only a reference to one commit. With its parental commits, the full branch structure can be constructed.
		\begin{itemize}
		\item Rapid branching and merging
		\item Tools for visualisation and navigation in development history
		\item Lightweight branches % branch = reference to only one commit
		\end{itemize}
	\item Distributed development
	%Like Darcs, BitKeeper, Mercurial, SVK, Bazaar, and Monotone, Git gives each developer a local copy of the full development history, and changes are copied from one such repository to another. These changes are imported as added development branches, and can be merged in the same way as a locally developed branch.
		\begin{itemize}
		\item Each developer has full history
			\begin{itemize}
			\item Prevents data loss
			\item Subteams can share reposities without access to central repository
			\end{itemize}
		\item No need to have access to central repository all the time
		\item Changes are commited locally and then pushed to central repository
		\end{itemize}%TODO anything else?
\end{itemize}
\end{frame}

\begin{frame}
\frametitle{Git characteristics}
\begin{itemize}
	\item Variety of protocols supported
	%Repositories can be published via Hypertext Transfer Protocol (HTTP), File Transfer Protocol (FTP), rsync (removed in Git 2.8.0[31]), or a Git protocol over either a plain socket, or Secure Shell (ssh). Git also has a CVS server emulation, which enables the use of extant CVS clients and IDE plugins to access Git repositories. Subversion and svk repositories can be used directly  with git-svn.
		\begin{itemize}
		\item HTTP/HTTPS
		\item FTP
		\item SSH %using passkey without need to provide username/pass
		\end{itemize}
	\item Efficient handling of large projects
	%Torvalds has described Git as being very fast and scalable,[32] and performance tests done by Mozilla[33] showed it was an order of magnitude faster than some version control systems, and fetching version history from a locally stored repository can be one hundred times faster than fetching it from the remote server.[34]
		\begin{itemize}
		\item Fast (when applying patches)
		\item Scalable
		\item Fetching version history from locally stored repository is faster then from remote
		\end{itemize}%TODO anything else?
	\item Allows various workflows
		\begin{itemize}
		\item Centralized (enterprise companies)
		\item Hierarchical (Linux kernel)
		\item Distributed (open source projects, pull requests)
		\end{itemize}%TODO anything else?
\end{itemize}
\end{frame}

% Ok, that was enough for theory
\subsection[]{Git locally}
\begin{frame}
\frametitle{Git Basics - commands}

\textbf{git init}
	\begin{itemize}
	\item Initializes empty local repository
	\end{itemize}
\textbf{git status}
	\begin{itemize}
	\item Shows current file differences between HEAD commit and current working copy
	\end{itemize}
\textbf{git add <filename>}
	\begin{itemize}
	\item Adds a file/directory into commit checklist
	\item -A (all files not versioned, or not ignored), -u (only updated files already under version control)
	\end{itemize}
\textbf{git commit -m <message>}
	\begin{itemize}
	\item Records working copy changes into repository
	\end{itemize}
\end{frame}

\begin{frame}
\frametitle{Git Basics - commands}

\textbf{git log}
	\begin{itemize}
	\item Shows latest commits for local repository
	\item -{}-oneline (condensed view), -{}-graph (includes branches)
	\end{itemize}
\textbf{git diff}
	\begin{itemize}
	\item Shows code difference between HEAD commit and current working copy
	\end{itemize}
\textbf{git checkout / git reset}
	\begin{itemize}
	\item Removes local uncommited changes
	\end{itemize}
\textbf{git reset -{}-soft HEAD~1 / -{}-hard <commithash>}
	\begin{itemize}
	\item Reverts working copy to given commit (soft keeps changes as \emph{to be commited}, hard removes them completely)
	\end{itemize}
\end{frame}

\begin{frame}
\frametitle{Git Basics - commands}
\textbf{git tag}
	\begin{itemize}
	\item Annotates current version of local repository with tag (such as version)
	\end{itemize}
\textbf{git clone}
	\begin{itemize}
	\item Clones remote repository into local repository and fetches latest changes
	\end{itemize}
\textbf{git push}
	\begin{itemize}
	\item Pushes local commited changes into remote repository
	\item -{}-tags Pushes tags into remote repository
	\end{itemize}
\textbf{.gitignore}
	\begin{itemize}
	\item File for specifying files not to be tracked under version control (binary files, log files, temporary build files, etc.)
	\end{itemize}
\textbf{.gitattributes}
	\begin{itemize}
	\item File for specifying attributes to apply for certain paths
	\item Used for example for specifying line endings
	\end{itemize}
\end{frame}




\subsection[]{Git Basics - Demo}
\begin{frame}
\frametitle{Git Basics - Demo}
\end{frame}





